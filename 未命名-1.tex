\documentclass[UTF8,12pt]{ctexart}
\usepackage{indentfirst}
\setlength{\parindent}{2em}
\renewcommand{\baselinestretch}{1.4}
\usepackage[a4paper]{geometry}
\geometry{verbose,
	tmargin=2cm,
	bmargin=2cm,
	lmargin=3cm,
	rmargin=3cm
}
\usepackage[x11names]{xcolor}
\usepackage{graphicx}
\usepackage{pstricks,pst-plot,pst-eps}
\usepackage{subfig}
\def\pgfsysdriver{pgfsys-dvipdfmx.def}
\usepackage{tikz}
\usepackage{verbatim}
\usepackage{url}

\title{一些日漫缩写的意义}
\author{destination}
\date{\today}

\begin{document}
\maketitle
\section{缩写}
\subsection{OVA (Original Video Animation)}
OVA(英语:Original Video Animation,原创动画录影带),原是指以录像带、影碟首次发行的动画影片,例如动画剧集或动画电影的续集或外传作品,如今也能通过影片分享网站的形式收看。

OVA是不在电视台上播放,也不在电影院放,直接贩卖的动漫

原创视频动画起初缩写为OAV,惟AV容易与影音(Audio and Video)及色情片(Adult Video)两词之缩写混淆。OVA之用语最早出现在角川书店的《Newtype月刊》上,以后并形成主流。目前仅有学习研究社的Animedia杂志、a1c. Co.公司的成人动画使用OAV之用语。

在2010年前后,因DVD与BD等光碟媒体已取代VHS等磁带媒体,亦有OAD(Original Animation Disc)之用语。该用语是讲谈社用于称呼单行本初回限定版中附赠的动画光碟,后泛指跟动画或小说一起发售的动画光碟。

相较于电视动画、剧场版电视或者电影院播放的不同,OVA则是从发行渠道来划分的,一般通过DVD等影碟的形式发行。一般的OVA不见得广为人知,选材一般是应某个特定作品的爱好者的要求而出的,一般是将情节补完,以满足爱好者收藏的需要。也有的OVA是作为实验期的作品,如果反响不错,就很有可能做成电视动画。


\subsection{CM}
\subsubsection{Commercial Message}
在日本,电视广告被称为“CM”,是「commercial message」(コマーシャルメッセージ)的简称,原本是指「商业用的讯息」,并没有特别限定是在电视上播出的。但在电视和广播的普及之下,渐渐成为电视广告的专用名词。而广义的「CM」则是包含电视、电影、网路上的广告影片,需要区别的时候,则改称「CF」(commercial film)。

\subsubsection{Comic Market}
由Comic Market准备会举办的日本乃至全球最大型的同人志即卖会,简称Comiket(コミケ)或 CM。一般夏Comi在每年八月的第二个星期五至星期日,冬Comi在十二月二十八及三十一日中的2-3天举办,并均在位于东京都江东区有名的东京国际展示场举行,免费入场。俗称御宅圣战、圣战。展会主要以动画、漫画、游戏、小说、周边的自费出版物的贩卖和展示为主要运营形式。

\subsubsection{Chinamofile}
把动画、漫画之类传到网络硬盘mofile上面,然后把提取码放在论坛上给别人分享。


\subsection{PV}
PV是promotion video(プロモーションビデオ)的意思 就是说是宣传用的录影带 一般正式发售的录影带为music video 。


\subsection{OP}
\subsubsection{OPEN}
OP通常是指OPEN简写,动漫的片头曲。

日本动画所谓的OP,全写Opening Song,日文名为オープニング曲,另有一较少人使用的名称为开头主题曲或动画片头曲,指动画开头所播放的歌曲。一般以每一个季度(13集)更换一套OP与ED。但总集数只有26集或更少则多数不会更换。一套动画中,OP数目如多于一首歌的,在随后发行的原声CD、网上讨论、以及动画资料整理等场合会以OP1,OP2,…等表示。部份人会把此术语套用于日本动画以外的动画,例如美国动画、欧洲动画等。
\paragraph{NCOP(Non-Credit Opening)}
就是没有字幕的OP动画

\subsubsection{One Piece}


\subsection{ED}
ED是END的简写,动漫的片尾曲。
日本动画所谓的ED,全写为Ending Song,日文名为エンディング曲,另有一较少人使用的名称为结尾主题曲或动画片尾曲,指动画在结尾所播放的歌。一般以每一个季度(13集)更换一套OP与ED。但总集数只有13集或更少则多数不会更改换。一套动画中,ED数目如多于一首歌的,在随后发行的原声CD、网上讨论、以及动画资料整理中等场合会以ED1,ED2,…等表示。根据统计,ED的数目往往比OP多。
\paragraph{NCED{Non-Credit Ending}}
就是没有字幕的ED动画


\subsection{SP}
SP就是英文缩写special也就是特别篇,附在动画前或后的特辑

\subsection{OVB}
OVB应该是特典的意思吧,这个不是很清楚


\subsection{OAD}
OAD是作为漫画的附赠品贩卖的日本的叫法


\subsection{Menu}
BD版本里的dvd菜单


\section{日本动画工业}
日本及其他国家的动画工业制作,大多向美国学习,流程和内容与美国动画工业没什么大差异。

\subsection{制前,又名前期制作}
企划、脚本(剧本)

人物设计、服饰设计

色彩设计、背景设计

音乐设计

机械设计、小物设定

\subsection{制作}
Layout、分镜表

作画:原画、中割(或称为中间画)

色彩设计、色彩指定

上色、摄影

配音、录音

\subparagraph{分镜}
分镜或分镜脚本,又称故事板(storyboard),是指电影、动画、电视剧、广告、音乐录影带等各种影像媒体,在实际拍摄或绘制之前,以故事图格的方式来说明影像的构成,将连续画面以一次运镜为单位作分解,并且标注运镜方式、时间长度、对白、特效等。
\subparagraph{原画}
原画是动画、电子游戏制作过程中由手绘的描述角色关键造型、动作的画,用于加工后期作品。


\subsection{制后,又名后期制作}
剪辑

试映

\subsection{发行}
电视动画

电视电影动画

剧场版动画

OVA

\section{基础概念}
\paragraph{ACGN}ACGN即日本动画(Anime)、漫画(Comic)与游戏(Game)的英文、轻小说(light Novel)首字母缩略字。该词汇通常不翻译成中文。
\paragraph{动漫}动漫是动画或漫画的合称与缩写,是在华人地区才有的称呼,另外西方国家将日本动画称Anime、漫画则称为Manga。而现今,动漫的发展已属于文化创意产业,同时是目前全世界热门且高人气的流行文化。
\paragraph{萌}萌(萌え)被用来做为对作品中的虚构角色表达强烈喜爱的用语,之后派生到可对各种事物表达类似情感。当作动词时,用于形容自己被某种人事物引发喜爱的情绪,例如“我被维基娘萌到了!”;当作形容词时,用来形容那些自己极端喜好的人事物,例如“维基娘好萌!”在中文网络用语的语境里,有时“萌”也会被引申为“可爱”的意思,或是将其作为“可爱”的同义词使用。
\paragraph{萌拟人化}指对非人类的动物、武器、国家或地区、交通工具、建筑物、食物等任何事物,依据其性质,把受仿物“描绘成具有与人相似的形体”和/或“演绎或变化成‘具有智能的形体’且‘可直接地或间接地表达若干原受仿物之特点属性与传奇’”形式的人型模样形容/设置/打扮。
\paragraph{御宅族}指热衷并熟悉动画、漫画及电子游戏的人群,后来扩大泛指热衷于次文化,或是对其有极深入了解的人们。研究御宅族的学问称为御宅学。也有贬义引申词“死宅”,多为御宅族自嘲用或是对御宅族的恶意贬低者使用。
\paragraph{同人}原指有着相同志向的人们、同好。后来转变成指“自创、不受商业影响的自我创作”,或“自主”的创作。它比商业创作有较大的创作自由度,以及“想创作什么,便创作什么”的意义。同人志则是这种创作的自制出版物,贩售同人志的活动称为同人志即卖会。这个文化圈则被称为同人界或同人圈。
\paragraph{控}控源自于日文コン的音译,而コン则是日语“情结”コンプレックス的简称。是一个后缀。在某种角色特征后面加上“控”,是指对该特征散发的吸引力怀有特殊情结,是一种带有偏执的喜好。例如父控、萝莉控等
\paragraph{Cosplay}指透过化妆、服饰、表演等方式装扮成虚拟角色的行为。从事Cosplay的人称为“Cosplayer”或简称“Coser”。“Cosplay”本身有时也简化为“Cos”使用。
\paragraph{变态}是用来形容带有色情或者是情色内容的作品;有时候也可以用来形容具有相似意义或者行为的一个词汇总称。同时也可以作名词使用,意指“喜爱变态作品的人”或是“会做出变态作品里某种行为的人”,但多带有贬义意味。
\paragraph{H}源自于“变态”一词日语读音罗马化后“Hentai”的首个字母“H”[1]。在中文ACGN文化里,H通常作为形容词表示“色情的”、“成人的”、“有黄色内容的”。在日语里,H有时也作动词代指“性交”(日本语:エッチ),也有不少中文字幕组将其翻译为中文后表示为“做色色的事情”。因其也可以理解为是汉字“黄”的中文发音“Huáng”的拼音首字母,所以在中文ACGN文化圈里也有着相当的使用频率。
\paragraph{工口}指情色或色情的要素。取日文エロ的字体而来,而エロ则是エロチック的简称,源自于英文的erotica。
\paragraph{次元}有两个涵义,一指作品当中的幻想世界以及其各种要素的集合体。例如,一个规则与秩序与读者现存的世界完全不同的世界经常被称为“异次元”;另指有别于现实三维空间的二维空间创作物,如动画及漫画中的虚拟角色被称为“二次元角色”。

\section{角色特征}
\subsection{性别}
\subsection{年龄}
\subsection{种族}
\subsection{身体}
\subparagraph{呆毛}指头发上的一根、一撮或多撮翘起的头发。在一些作品中,呆毛会随着主人的心情而变形,或是当作雷达、武器使用
\subparagraph{欧派}指胸部、乳房,音译自日语:おっぱい(罗马化:Oppai)。
\subsection{服饰}
\subparagraph{死库水}日本学生泳衣(スクール水着)简称“スク水”的音译。
\subparagraph{胖次}内裤(パンツ)的日语音译。
\subparagraph{女子高校生三神器}指水手服、女性学生泳衣和灯笼裤三项女高中生制服的合称。
\subparagraph{绝对领域}指穿着迷你裙与膝上袜时大腿暴露出来的部分。
\subsection{性格}
\subparagraph{病娇}指处于精神疾病的状态下和其他人发展出爱情,或对爱情的执著非比寻常到足以视为精神问题的一种性格。
\subsection{行为}
\subparagraph{壁咚}指一人将另一人逼向墙壁的同时,以手抵住墙面围住对方的动作。因抵住墙壁时或被围之人后退时碰到墙壁往往发出“咚”的声音而得名。
\subparagraph{颜艺}指角色因为某些原因使得脸部的表情极度扭曲

\section{人际关系}
\paragraph{配对}
又称为CP,是指二次创作中角色之间的恋爱关系。通常用“某甲×某乙”来表示,中间的“×”不发音,在×前者表示攻、在×后者表示受
\paragraph{NTR}
日语“寝取られ”的缩写,是指自己的伴侣遭人睡走,近似于中文的戴绿帽子
\subsection{职业、身份}
\subsection{性取向}
\subparagraph{BL}
指虚构幻想的男同性恋作品。
\subparagraph{蔷薇}
指虚构的男性间爱慕关系。
\subparagraph{GL}
指虚构幻想的女同性恋作品。
\subparagraph{百合}
指虚构的女性间爱慕关系。


\section{产业}
\subsection{从业人员}
\subparagraph{监督}在日语,监督泛指管理、指导的负责人。在动画中指的是导演,也就是集成全部艺术元素的艺术生产负责人。
\subparagraph{声优}(又称character voice,简称cv)- 即是配音员,指在动画、游戏、电影、有声书里帮角色配音的人员。
\subsection{作品类型}
\subsubsection{番}源自于日文的番組,意思是“节目”,在动漫迷间又特指动画节目
\subparagraph{新番}指当季推出的最新动画,过了当季后会被称为旧番
\subparagraph{季番}指播放区间为一季(三个月)的动画,集数通常为12集。播放区间为半年者称半年番,为一年者称年番
\subparagraph{表番}又称为一般向,用以称呼非限制级作品
\subparagraph{里番}相对于表番,用以称呼限制级作品
\subparagraph{肉番}又称肉片。指特别重视展现色情向度而甚于角色塑造、剧情深度的作品
\subparagraph{泡面番}指时间很短的连载作品,由于其时长接近一份泡面的准备时间而得名
\subparagraph{番外}指外传,以原本的故事为底的额外故事
\subparagraph{OVA}原是指以录像带、影碟首次发行的动画影片,例如动画剧集或动画电影的续集或外传作品,如今也能通过视频分享网站的形式收看
\subparagraph{定番}指常见甚至是固定的桥段。可能是经典的、也可能是老套的
\subsubsection{男性向}指以男性族群为诉求对象的作品。
\subsubsection{女性向}指以女性族群为诉求对象的作品。
\subsubsection{子供向}指以幼儿、儿童族群为诉求对象的作品
\subsubsection{R18}即18禁作品,18岁及以上年龄段才能观赏的限制级作品(此类作品多含情暴力、情色情节或场景)
\subsubsection{全年龄向}指能够面对全部年龄段的观众的作品(与R18相反,不含任何暴力、情色情节或场景)
\subsubsection{后宫型作品}指的是一个男性主角和至少两个以上的数个、十数个甚至更多的女性角色有暧昧、爱慕或性关系的作品。
\subsubsection{逆后宫型作品}指的是一个女性主角和至少两个以上的数个、十数个甚至更多的男性角色有暧昧、爱慕或性关系的作品。
\subsubsection{治愈系}指观赏后能使人内心平静、感受到心灵慰藉的作品
\subsubsection{致郁系}与“治愈系”作品相反,观看后无法使人平静,甚至会导致抑郁。此类作品分为两类:一类为设置或情节黑暗,例如《魔法少女小圆》;另一类作品情节悲情、催泪,例如《CLANNAD》
\subsubsection{空气系}指以人物的平凡日常生活为主题的作品,内容节奏缓慢、也没有沉重情节
\subsubsection{世界系}是指以主角与他人为中心的、微小的关系性问题,与世界的危机及世界的灭亡等抽象的大问题直接联系的作品
\subsubsection{废萌}指仅重视展现萌属性而忽略角色塑造、剧情深度的作品
\subsubsection{无印}指系列作中为了区别有副标题的后续作品,对没有副标题的首部作品的称呼。例如在《神奇宝贝》动画中,为了与2002年至2006年播出的《神奇宝贝超世代》、2006年至2010年播出的《神奇宝贝钻石+珍珠》及后续的动画做区别,而将1997年至2002年播出的、没有副标题的《神奇宝贝》动画称为无印。
\subsubsection{猎奇}指作品内容包含血腥、残酷、惊悚或其它普遍而言令人不悦的元素
\subsubsection{萝卜}指机器人题材作品,为英语“robot”的音译
\subsubsection{Yaoi(やおい)}是指以虚构男同性恋爱情为题材的作品
\subsubsection{百合}是指以虚构女同性恋爱情为题材的作品
\subsubsection{鬼畜}原本为佛教对六道中饿鬼道与畜生道之合称,此处用来指非人道或邪恶的性格,言行。近来亦引申用来形容造成精神猎奇效果的二次元事物
\subsection{制作术语}
\subparagraph{总集篇}是指动画中将多集的部分画面截取、重新剪辑成的一集。总集篇可以概括地介绍剧情,也可以在制作时间不够时用来争取空档
\subparagraph{腰斩}指原本预期持续进行中的作品在中途因故结束
\subparagraph{杀必死}在原本的故事剧情以外,提供增加作品的娱乐性以吸引观众的内容。亦常用以指多少刻意的,具有情色意味的镜头。源自英文单字“service”的发音
\subparagraph{flag}是在故事中表示事情将引发以后特定发展或状况的用语。与伏笔的涵义类似,但用来指较为单纯有固定模式的表现
\subparagraph{声优之盾}对日本动画或电子游戏等制作团队在决策错误时请配音员(声优)道歉的行为的蔑称
\subparagraph{圣光}在部分成人动画或成人漫画中,为了符合相关法律规范,会将露出生殖器的画面或过于裸露的画面以涂白方式遮蔽,此种作法被戏称为“圣光”。


\section{爱好者及其行为}
\subsection{爱好者}
\subparagraph{腐人类}指喜欢虚构男同性恋间爱情的人,女性称为“腐女”、男性则称为“腐男”
\subparagraph{同人女}原指进行同人志创作的女性群体,后因为进行同人创作的女性大多喜爱耽美类同人作品,所以现在这一词汇经常被误用为泛指创作与欣赏耽美作品的女性
\subparagraph{历女}意指喜爱历史的女性
\subparagraph{兽迷}是指喜好具有人格或其他人类特质的拟人化动物之虚构角色的一群人
\subparagraph{御宅艺}是一种由御宅族或日本偶像支持者表演的舞蹈或打气动作,其中包括跳跃、拍掌、挥动手臂和有节奏地喊口号
\subparagraph{萌豚}对以二次元美少女为卖点的ACGN作品情有独钟的阿宅,该词从创造便带有贬义性质
\subparagraph{实况}原指比赛中报导赛况的播报员,此指以摄影机和麦克风等设备将自己在观看动画或游玩电子游戏时的影音以现场直播或预先录影的形式上传至视频分享网站。如果使用文字,也可称为实况。进行实况的人称为实况主
\subparagraph{up主}意指在视频网站、资源网站等地上传视频、音频或其他文件的人(即投稿人),华语圈中常指(但不限于)弹幕视频网站中的投稿人

\subsection{行为}
\subparagraph{追番}指积极地观看某个新作品的行为,如果在途中因为对作品没有兴趣而停止观看,则称为弃番
\subparagraph{推坑}指某作品的强烈爱好者向其周遭的人宣传该作品,并希望对方也能喜欢该作品的行为
\subparagraph{neta}又名捏他、梗,指故事情节中的最重要部分,派生词有剧透(ネタバレ),即透露作品剧情的行为
\subparagraph{痛}把喜欢的角色表现在车辆、衣服上的行为,或是指这样的物品
\subparagraph{把灵魂卖给路西法}是指看到、或者热爱伪娘作品;以及爱上了伪娘的代名词。源自于一位网民在看完了日本成人动画《Pico系列》后的反应。影片中的句子如“Yooooo”、“THREEEEEE”等也一并走红
\end{document}